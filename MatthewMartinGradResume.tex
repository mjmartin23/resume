% Matthew J. Martin
% June 20, 2016
% Resume

\documentclass{myresume}

\usepackage[hidelinks]{hyperref}
\usepackage{nopageno}
\usepackage[tmargin=1in,left=1in,right=1in,bmargin=1in]{geometry}

\begin{document}

\name{Matthew J. Martin}
% \vspace{-1em}
\contact{3901 1st Ave. NW Unit 205, Seattle, WA 98107}{\href{mailto:mattjmrtn@gmail.com}{mattjmrtn@gmail.com}}{(774)-392-5741}

% \vspace{-1.5em} % bring section header up
\section{Education}
\textbf{Colby College}, Waterville, ME\hfill\textbf{Bachelor of Arts, May 2018}

\textit{Majors}: Government and Computer Science, \textit{Minor}: Mathematics\hfill\textbf{Overall GPA: 3.83/4.00}

\textit{Honors}: Dean's List: Fall 2014, Fall \& Spring 2015, Fall \& Spring 2017

\textit{Relevant Coursework}: Robotics, Introduction to Vision and Robotics, Object-Oriented Systems,\\Software Design and Modelling, Data Structures and Algorithms, Data Analysis and Visualization,\\Parallel and Distributed Processing, Computer Organization, Linear Algebra, Statistics

\section{Programming Experience and Projects}

\projecttype{Research}
\project{Better Predictors for Issue Lifetime}{Lead Researcher}{2016}{
	\item I developed a simpler and more accurate method of predicting issue lifetime in software projects than the current state of the art using machine learning optimizations and  careful feature selection
	\vspace{-.5em}
	\item I co-authored a paper documenting this finding, which is currently under review by the Journal of Systems and Software
	}

\projecttype{Robotics}
\project{Multi-floor Wheeled Robot}{Co-Developer}{2017}{
	\item This wheeled robot is able to access multiple floors of a building by finding an elevator with the help of a person
	\vspace{-.5em}
	\item The process involves human-robot interaction, face recognition, color following, and line detection to achieve its goal
}
\project{Line Following Robot}{Co-Developer}{2016}{
	\item A robot made to accomplish a few tasks involving line following, implemented in Python using a PID controller
}

\projecttype{Computer Vision/Graphics}
\project{Augmented Reality Chess}{Sole Developer} {2017} {
	\item An AR chess game written in C++ that allows two users to play the game by physically selecting and moving virtual pieces while the entire board and pieces are projected digitally on to a video stream
}
\project{Coin Counter}{Co-Developer}{2016}{	
	\item A Matlab program that takes in an image of several small objects (including coins) on a tabletop, segments and classifies each object in the scene, and outputs the total amount of money present
}

\projecttype{Java Development}
\project{Bantam Java Compiler and Optimizer}{Team Member}{2017}{
	\item A compiler and optimizer written from scratch for compiling a subset of the Java language, called Bantam Java
	\vspace{-.5em}
	\item Worked on a team of four, focusing on software design principles, using the Visitor pattern and other OO strategies
}
\project{Easy Calculator}{Sole Developer}{2015}{
	\item A user-friendly calculator app for Android devices made with modern design in mind}

%\project{Advanced Data Visualization}{Co-Developer}{2016}{	
%	\item A fully functional data visualization application written in Python that uses object tracking to allow the user to pan, scale, and rotate the data simply by moving their hands in front of the screen
%}

%\project{Music LED Controller}{Sole Developer}{2015}{
%	\item Analyzes music played through a Raspberry Pi, displaying a visualization of the frequencies via an LED strip
%}

%\projecttype{Web Development}
%\project{Colby College Computer Science Department Website}{Co-Developer}{2016}{
%	\item Redesigned and now maintaining the official department website for Colby CS: \href{http://cs.colby.edu}{cs.colby.edu}
%	\vspace{-.5em}
%	\item Rebuilt the site from the ground up, and it is now accessed by every computer science student at Colby on a daily basis
%	}
%\project{Personal Site}{Sole Developer}{2016}{
%	\item The domain \href{http://matthewmartin.me}{matthewmartin.me} is consistently updated with recent work, and has  detailed descriptions of my projects
%}
	

%\project{Easy Flashlight}{Sole Developer}{2014}{
%	\item An easy to use flashlight for Android devices
%}


Source code for most projects is available on \textbf{GitHub} at: \href{https://github.com/mjmartin23}{github.com/mjmartin23}

\section{Coding Skills}

\begin{minipage}[t]{0.33\textwidth}
	\textbf{Skilled in}:
	\begin{itemize}
		\vspace{-.5em}\item Java
		\vspace{-.75em}\item Python
		\vspace{-.75em}\item C/C++
		\vspace{-.75em}\item HTML/CSS
	\end{itemize}
\end{minipage}
\begin{minipage}[t]{0.33\textwidth}
	\textbf{Experience with}:
	\begin{itemize}
		\vspace{-.5em}\item Git
		\vspace{-.75em}\item Matlab
		\vspace{-.75em}\item UML
		\vspace{-.75em}\item Unix shell
	\end{itemize}
\end{minipage}
\begin{minipage}[t]{0.33\textwidth}
	\textbf{Familiar with}:
	\begin{itemize}
		\vspace{-.5em}\item Javascript/jQuery
		\vspace{-.75em}\item MySQL
		\vspace{-.75em}\item PHP
		\vspace{-.75em}\item Linux, Windows
	\end{itemize}
\end{minipage}

\section{Employment}

\job{Software Developer Engineer II}{Amazon}{September 2018 - present}{
	\item Worked on a team responsible for creating a virtual model of
	every item that Amazon sells
	\item 
}

\job{Software Development Engineering Intern}{Amazon}{May - August 2017}{
	\item Worked with experienced engineers developing Amazon's ``vision tunnels" image processing algorithm for sorting packages
}
\job{Undergraduate Researcher}{North Carolina State University}{May - August 2016}{
	\item Worked with \href{http://menzies.us}{Dr. Tim Menzies} researching hypotheses from industrial partners
}
\job{Teaching Assistant}{Colby College Computer Science Department}{February 2015 - present}{
	\item Assisted students with their computer science projects during and outside of class
}

\section{Activities}

\activity{Member}{Colby Hackers}{2015 - present}
\activity{Member}{Colby College Class Council}{2015 - present}
\activity{Member}{Colby College Men's Tennis}{2014 - present}
	
\end{document}